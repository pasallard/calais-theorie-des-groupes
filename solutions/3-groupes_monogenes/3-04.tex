\begin{enumerate}[a)]
  \item Soit $n\in\N^*$ et $a\in\Z^*$ premier avec $n$. Alors $a$ est un élément inversible de l'anneau $\grq{\Z}{n\Z}$, \textit{i.e.} du groupe $\Z^\times_n$ dont le cardinal est $\varphi(n)$ (proposition 3.24). Alors, d'après le corollaire (2.10) du théorème de Lagrange, on a $\overline{a}^{\varphi(n)}=\overline{1}$
    dans $\Z^\times_n$, c'est-à-dire $a^{\varphi(n)}\equiv 1\pmod{n}$.
  
  \item Soit $p$ est un nombre premier et $a\in\Z$. Si $p$ ne divise pas $a$
    alors d'après le théorème d'Euler $a^{\varphi(p)}\equiv 1\pmod{p}$.  Comme
    $p$ est premier, $\varphi(p)=p-1$ donc $a^{p-1}\equiv 1\pmod{p}$ puis
    $a^p\equiv a\pmod{p}$. Si $p$ divise $a$ alors $a^p\equiv a\pmod{p}$ est
    évident. Le théorème de Fermat est démontré.
\end{enumerate}


