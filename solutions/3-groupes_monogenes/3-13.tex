Soit $C_n=\langle x \rangle$ un groupe cyclique d'ordre $n>1$, et $k$ un entier
tel que $1\leqslant k\leqslant n-1$. Soit $d = \pgcd(k,n)$. Nous avons les
équivalences
%
\begin{align*}
  (x^k)^l = e
  &\iff x^{kl} = e \\
  &\iff n \divise kl && \text{(corollaire 3.7)} \\
  &\iff \frac{n}{d} \divise \frac{k}{d}l
    && \text{où $\pgcd\Bigl(\frac{n}{d},\frac{k}{d}\Bigr) = 1$}\\
  &\iff \frac{n}{d} \divise l && \text{(lemme de Gauss).}
\end{align*}
%
Ainsi l'ordre de $x^k$ est égal à $\dfrac{n}{d}$.

\begin{remarque}
  On retrouve comme cas particulier la propriété énoncée page 99 : $x^k$ $(0< k<n)$ est un générateur du
  groupe $C_n$ si et seulement si $k$ et $n$ sont premiers entre eux.
\end{remarque}
