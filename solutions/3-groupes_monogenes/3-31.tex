% Pierre-Alain Sallard
\emph{Résolution succincte}

On adopte la même démarche que dans le cours (page 122 et suiv.). Étant donné $\varphi \in S_{\R}(\Z)$, on note $k = \varphi(0) \in \Z$. Puisque $\varphi$ conserve les distances, $\varphi(1) = k \pm 1$. Si $\varphi(1) = k + 1$, on a nécessairement $\varphi = \tau_k= \tau_1^k$. Et si $\varphi(1) = k - 1$, on montre que  $\varphi = \tau_{1}^k\circ \sigma_0$. CQFD.
