\begin{enumerate}
  \item
    On reprend l'application $\varphi$ de l'exercice 5, dont on sait déjà qu'elle est surjective. On note que la minimalité de $\set{a_1,a_2,\dots,a_r}$ implique que $a_i \neq e$ pour tout $i$, donc chaque élément $a_i$ est d'ordre~2.
    
    \emph{Existence.} 
    Avec les notation de l'exercice 5, chaque $x_i$ de $\Gr{a_i}$ s'écrit $a_i^{k_i}$ avec $k_i \in \{0,1\}$ puisque $o(a_i) = 2$. D'où l'existence de l'écriture recherchée.
    
    \emph{Unicité.} On va montrer que $\varphi$ est injective. On montre aisément que c'est un morphisme de groupes donc il suffit d'étudier son noyau. Soit $y=(a_1^{k_1},\ldots, a_r^{k_r}) \in \Ker \varphi$. Puisque $a_1^{k_1}\times \cdots \times a_r^{k_r} = e$ et que $a_1$ est d'ordre 2, il vient que $a_2^{k_2}\times \cdots \times a_r^{k_r}= a_1^{k_1}$. Si on avait $k_1=1$, alors $a_1$ appartiendrait au groupe engendré par la famille $(a_2, \ldots, a_r)$, ce qui contredirait l'hypothèse de minimalité de la famille génératrice de $G$. Donc $k_1 = 0$. Et on répète cet argument pour tous les autres $k_i$ (car $G$ est abélien), et il vient que $y = a_1^0\times\cdots\times a_r^0 = e$. Ainsi $\varphi$ est bien injective, ce qui fournit l'unicité de l'écriture recherchée.
  
  \item
     Puisque chaque élément $a_i$ est d'ordre~2, on a $\Gr{a_i}\iso C_2$. Et $\varphi$ est un isomorphisme de groupes d'après la question précédente : nous en déduisons que $G$ est isomorphe à $C_2^r$ et concluons que le groupe $G$ est d'ordre $2^r$.
\end{enumerate}
