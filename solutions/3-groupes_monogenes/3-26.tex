% Pierre-Alain Sallard

\begin{enumerate}

 \item On vérifie classiquement les décompositions du texte.

 \item \begin{enumerate}

 \item Une permutation $\sigma \in A_n$ est le produit d'un nombre pair de transpositions : on peut donc regrouper par paires ces transpositions. Or chacune des paires est soit de la forme $(i,j)(j,k) = (i,j,k)$, soit de la forme $(i,j)(k,l)=(i,j,k)(j,k,l)$ donc on peut bien écrire $\sigma$ comme un produit de 3-cycles.
 \item Il suffit d'établir (proposition 1.38) qu'un 3-cycle $(i,j,k)$ peut s'écrire comme un produit de 3-cycles $(1,a,b)$ ou de leurs inverses $(1,a,b)^{-1} = (b,a,1)$. On montre que $(i,j,k) = (1,i,k)^{-1}\circ (1,i,j)$, d'où le résultat.
 \item Il suffit de montrer qu'un 3-cycle de la forme $(1,i,j)$ peut se décomposer à l'aide de 3-cycles de la forme $(1,2,a)$ ou de leurs inverses. On établit que $(1,i,j) = (1,2,i)^{-1}\circ (1,2,j)^{-1}\circ (1,2,i)$, d'où le résultat.
\end{enumerate}
\end{enumerate}

