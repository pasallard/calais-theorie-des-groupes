Le groupe $G$ opère par translation à gauche sur l'ensemble quotient $Q_H = \smash{\clsg{G}{H}}$.

\emph{Méthode 1 :}
Considérons le morphisme de groupes $\gamma\from G\to\Sym{Q_H}$ associé à cette action et notons $K$ son noyau. 
Il est normal dans $G$ et est inclu dans $H$ (proposition~5.15).
Le groupe $\grq{G}{K}$ est isomorphe à un sous-groupe de $\Sym{Q_H}$ (1\ier{} théorème d'isomorphisme), donc $[G:K]$ divise $\card{\Sym{Q_H}} = [G:H]! = n!$.

\emph{Méthode 2 :} on va appliquer le résultat de l'exercice 10, question c), au cas où $x =eH$ de sorte que $X = \Omega_x = \set{gH \given g\in G} = Q_H $. Avec les notations de cet exercice, $G_X = \set{k\in G\given \forall g\in G,~ kgH = gH} = \set{k\in G\given \forall g\in G,~ k\in gHg^{-1}} = \bigcap_{g\in G} gHg^{-1}$. On a clairement $G_X\subset H$ et on sait déjà que $G_X\normal G$, avec $\grq{G}{G_X}$ isomorphe à un sous-groupe de $\Sym{X}= \Sym{Q_H}$ ce qui induit que $[G:G_X]$ divise $\card{\Sym{Q_H}} = [G:H]! = n!$. Le sous-groupe $G_X$ à donc bien les propriétés attendues.
