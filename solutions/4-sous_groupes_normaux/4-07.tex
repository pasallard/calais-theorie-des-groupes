% Pierre-Alain Sallard
\begin{enumerate}
 \item
 \begin{itemize}
  \item  Soit $\cl{x} \in \overline{G}$ et $x$ un de ses représentants dans $G$. Puisque $S$ engendre $G,~ \exists n\in \N^*,~\exists x_1,\ldots, x_n\in S$ tel que $x = \prod\limits_{i=1}^n x_i$. Alors $\cl{x} = \prod\limits_{i=1}^n \cl{x_i}$, ce qui justifie que $\overline{S}$ en gendre $\overline{G}$.
  \item Si $S =\{x_1,\ldots, x_m\}$, alors $\overline{S} \subset  \{\cl{x_1},\ldots, \cl{x_m}\}$ d'où le résultat.
 \end{itemize}

  \item $\bullet$ Soit $x\in G:~ \cl{x} \in \overline{G}$ donc $\exists n\in \N^*,~\exists \cl{x_1},\ldots, \cl{x_n}\in \overline{S}$ tel que $\cl{x} = \prod\limits_{i=1}^n \cl{x_i}$. Puisque $\forall y\in G,~ \cl{y} = yH,~ \cl{x} = \cl{x_1\ldots x_n} = \left(\prod\limits_{i=1}^nx_i\right)H$. Puisque $x\in \cl{x},~ \exists h\in H,~ x = \left(\prod\limits_{i=1}^n x_i\right)h$. Et $T$ engendre $H\implies \exists r\in\N^*,~ \exists t_1,\ldots, t_r$ tel que $h = \prod\limits_{j=1}^r t_j$. D'où $x = \left(\prod\limits_{i=1}^n x_i\right)\left(\prod\limits_{j=1}^r t_j \right) \in \Gr{S\cup T}$. CQFD.

 $\bullet$ Le deuxième point vient facilement en reprenant la démarche ci-dessus.
\end{enumerate}

