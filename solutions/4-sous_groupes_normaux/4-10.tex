% Pierre-Alain Sallard
\begin{enumerate}
 \item Les douze éléments de $A_4$ sont l'identité $e$ et les compositions de deux transpositions :
 \begin{itemize}
  \item les 3 permutations d'ordre 2 : $(1,2)(3,4),~ (1,3)(2,4)$ et $(1,4)(2,3)$ ;
  \item et les 3-cycles (d'ordre 3) $(1,2,3),~ (1,3,2),~ (1,2,4),~ (1,4,2),~ (1,3,4),~ (1,4,3),~ (2,3,4),~ (2,4,3)$.
 \end{itemize}
\item L'exercice 14 du chapitre III montre que, si $x,y$ commutent dans $G$ d'ordre fini avec $\pgcd(\ordre(x), \ordre(y))=1$, alors $\ordre(xy) = \ordre(x)\times \ordre(y)$. Par l'absurde, on suppose que $\ZG(A_4)\neq \{e\}$ et on se donne $x\neq e \in \ZG(A_4)$. Puisque $\ordre(x)\in \{2,3\},~ \exists y\in A_4,~ \ordre(y) = \frac{6}{\ordre(x)}$ (si $x$ est une permutations d'ordre 2, on on prend pour $y$ l'un des 3-cycles,et inversement). Mais $x$ commute avec $y$ car $x\in \ZG(A_4)$ donc $\ordre(xy) =6$. Mais il n'existe aucun élément d'ordre 6 dans $A_4$, d'où la contradiction. Ainsi,~ $Z(A_4) = \{e\}$.
\item On a vérifié à l'exercice 9 que $K:=\{e,(1,2)(3,4),~ (1,3)(2,4),~(1,4)(2,3) \}$ est un sous-groupe (d'ordre 4) de $A_4$ et qu'il est normal dans $\Sym{4}$, donc \textit{a fortiori} dans $A_4$. Et il ne peut pas y avoir d'autre sous-groupe de $A_4$ d'ordre 4 : en effet, les éléments d'un tel sous-groupe ont un ordre qui divise 4 et cela exclut tous les 3-cycles de $A_4$. Par élimination, il n'y a que $K$ qui convient.
\item On suit l'indication en considérant d'abord $K\cap K'$ : suppose $\exists x\neq e\in K\cap K'$. Alors $\ordre(x)\neq 1$ divise 4 et 6 donc $\ordre(x) = 2$. Le sous-groupe $\Gr{x}$ est d'ordre 2 et il est normal dans $A_4$ (car $x\in K\normal A_4$) donc, d'après l'exercice 1 de ce chapitre, $\Gr{x} \subset \ZG(A_4)$. Or ceci contredit le résultat de la question 2, donc $K\cap K'=\{e\}$.

Compte-tenu de la définition de $K$, ceci implique que $K'$ ne contient, outre $e$, que des 3-cycles. Or, si $\sigma \in K'$ est un 3-cycle, $\sigma^2 \neq \sigma$ appartient aussi à $K'$ : le nombre de 3-cycles appartant à $K'$ est donc pair.
En y ajoutant l'identité $e$, le cardinal de $K'$ serait alors impair, ce qui exclut la possibilité d'avoir $\ordre(K') = 6$. CQFD.
\end{enumerate}
