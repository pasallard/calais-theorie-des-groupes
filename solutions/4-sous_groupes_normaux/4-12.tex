% Pierre-Alain Sallard

$\Gamma_{\infty}$ est clairement un sous-groupe propre de $\U$, isomorphe à $\grq{\Q}{\Z}$ par application du premier théorème d'isomorphisme à la restriction à $\Q$ du morphisme $\varphi \colon (\R,+)\to (\U,\times)$ de l'exercice 11, point a-2, où $\forall x\in \R,~ \varphi(x) = e^{\I 2\pi x}$.

On avait établi dans ce précédent exercice l'existence de l'isomorphisme $\overline{\varphi}\colon \grq{\R}{\Z}\to \U$. Il s'avère avec le point précédent que $\overline{\varphi}\left(\grq{\Q}{\Z} \right) = \Gamma_{\infty}$ : le lemme 4.32 (et la remarque 4.33) assure l'isomorphisme entre les groupes quotients $\grq{\R/\Z}{\Q/\Z}$ et $\grq{\U}{\Gamma_{\infty}}$. Mais par le troisième théorème d'isomorphisme 4.36,
$\grq{\R/\Z}{\Q/\Z}\iso \grq{\R}{\Q}$ donc on a bien $\left(\grq{\R}{\Q},+\right)\iso \left(\grq{\U}{\Gamma_{\infty}},\times\right)$.


