%%% Exercice 4.19, ajout de Pierre-Alain Sallard

Méthode 1, par approche directe. Soit $g\in G$ et $z\in Z(H)$ : posons $x = gzg^{-1}$ et il s'agit de montrer que $x\in Z(H)$, c'est-à-dire qu'il commute avec tout élément de $H$.
Soit alors $h\in H :~ xh = gzg^{-1}h = gz(g^{-1}hg)g^{-1}$. Mais puisque $H \normal G,~ g^{-1}hg \in H$ donc commute avec $z$. Ainsi $x = g (g^{-1}hg)zg^{-1} = h gzg^{-1} = hx$. CQFD.


Méthode 2, à base de théorèmes du cours.
Nous avons $H\normal G$ et $\ZG(H)\caracteristique H$ (proposition~4.43), donc $\ZG(H)\normal G$ (proposition~4.44).

Dans le groupe $Q_8$ (exercice~4.18), considérons le sous-groupe normal $H = \Gr{a}$ : il est cyclique, donc abélien donc $\ZG(H) = H = \{e, a, a^2, a^3\}$. Mais comme $\ZG(Q_8) = \{e, a^2\}$, l'inclusion $\ZG(H) \subseteq \ZG(Q_8)$ est exclue.
